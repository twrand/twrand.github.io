\documentclass[letterpaper,11pt,twoside]{article}

\usepackage[left=1in, right=1in, bottom=1.25in, top=1.5in]{geometry}
\usepackage[utf8]{inputenc}
\usepackage{setspace}
%\usepackage{hyperref}
\usepackage{fancyhdr}
\usepackage{amsmath}
\usepackage{amsfonts}
\usepackage{amssymb}
\usepackage{amsthm}
\allowdisplaybreaks
\usepackage[T1]{fontenc}
\usepackage{xcolor}
\usepackage[mathscr]{euscript}
\usepackage{latexsym,bbm,xspace,graphicx,float,mathtools,mathdots,xspace}
\usepackage{enumitem}
\usepackage[ruled,vlined]{algorithm2e}
\usepackage{bm}
\usepackage[backref, colorlinks,citecolor=blue,linkcolor=magenta,bookmarks=true]{hyperref}
\usepackage[nameinlink]{cleveref}

% tikz
\usepackage{subfig}
\usepackage{graphicx}
\usepackage{tikz}

\usepackage{tablefootnote}

\fancypagestyle{plain}{%
\fancyhf{} % clear all header and footer fields
\fancyfoot[C]{\textbf{\thepage}} % except the center
\renewcommand{\headrulewidth}{0pt}
\renewcommand{\footrulewidth}{0pt}}

\theoremstyle{plain}
\newtheorem{theorem}{Theorem}
\newtheorem{assumption}{Assumption}
\newtheorem{corollary}{Corollary}
\newtheorem{lemma}{Lemma}
\newtheorem{conjecture}{Conjecture}
\newtheorem{proposition}{Proposition}
\newtheorem{observation}{Observation}
\newtheorem{claim}{Claim}
\newtheorem{property}{Property}
\newtheorem{op}{Open Problem}
\newtheorem{problem}{Problem}
\newtheorem{question}{Question}

\theoremstyle{definition}
\newtheorem{definition}{Definition}
\newtheorem{example}{Example}
\newtheorem{sketch}{Sketch}
\newtheorem{idea}{Idea}

\theoremstyle{remark}
\newtheorem{remark}{Remark}

\newtheoremstyle{restate}{}{}{\itshape}{}{\bfseries}{~(restated).}{.5em}{\thmnote{#3}}
\theoremstyle{restate}
\newtheorem*{restate}{}



\crefname{theorem}{Theorem}{Theorems}
\crefname{assumption}{Assumption}{Assumptions}
\crefname{corollary}{Corollary}{Corollaries}
\crefname{lemma}{Lemma}{Lemmas}
\crefname{conjecture}{Conjecture}{Conjectures}
\crefname{proposition}{Proposition}{Propositions}
\crefname{observation}{Observation}{Observations}
\crefname{claim}{Claim}{Claims}
\crefname{property}{Property}{Properties}
\crefname{op}{Open Problem}{Open Problems}
\crefname{problem}{Problem}{Problems}
\crefname{question}{Question}{Questions}

% \crefname{fact}{Fact}{Facts}

\crefname{definition}{Definition}{Definitions}
\crefname{example}{Example}{Examples}
\crefname{sketch}{Sketch}{Sketches}
\crefname{idea}{Idea}{Ideas}

% \crefname{condition}{Condition}{Conditions}

\crefname{remark}{Remark}{Remarks}



\crefname{equation}{Equation}{Equations}
\crefname{figure}{Figure}{Figures}
\crefname{table}{Table}{Tables}



%%%%%%%%%%%%%%%%%%%%%%%%%%
% GENERAL-PURPOSE MACROS %
%%%%%%%%%%%%%%%%%%%%%%%%%%
\newcommand{\uth}{\bigskip \bigskip {\huge {\red{UP TO HERE}} \bigskip \bigskip}}
\newcommand{\ignore}[1]{}
\newcommand{\eps}{\varepsilon}
\newcommand{\simple}{\mathrm{simple}}
\newcommand{\E}{\operatorname{{\bf E}}}
\newcommand{\Ex}{\mathop{{\bf E}\/}}
\renewcommand{\Pr}{\operatorname{{\bf Pr}}}
\newcommand{\Prx}{\mathop{{\bf Pr}\/}}
\newcommand{\Var}{\operatorname{{\bf Var}}}
\newcommand{\Varx}{\mathop{{\bf Var}\/}}
\newcommand{\tO}{\tilde{O}}
\newcommand{\sgn}{\mathrm{sgn}}
\newcommand{\X}{\mathcal{X}}
\newcommand{\Y}{\mathcal{Y}}
\newcommand{\rZ}{\mathcal{Z}}

\DeclareMathOperator\erf{erf}

\newcommand{\polylog}{\mathrm{polylog}}
\newcommand{\poly}{\mathrm{poly}}
\newcommand{\bcalZ}{\bm{\mathcal{Z}}}
\newcommand{\bx}{\bm{x}}
\newcommand{\by}{\bm{y}}
\newcommand{\bxi}{\bm{\xi}}

%%%%%%%%%%%%%%%%%%
% NUMBER SYSTEMS %
%%%%%%%%%%%%%%%%%%
\newcommand{\R}{\mathbb R}
\newcommand{\RR}{\R_{\geq 0}}
\newcommand{\C}{\mathbb C}
\newcommand{\N}{\mathbb N}
\newcommand{\NN}{\N_{\geq 1}}
\newcommand{\Z}{\mathbb Z}

\renewcommand{\i}{\mathbf{i}}   % for complex numbers
\renewcommand{\d}{\mathrm{d}}   % for integrals
\newcommand{\lhs}{\mathrm{LHS}} % for inequalities
\newcommand{\rhs}{\mathrm{RHS}} % for inequalities
\newcommand{\supp}{\mathrm{supp}}
\renewcommand{\hat}[1]{\widehat{#1}}
\renewcommand{\bar}[1]{\overline{#1}}
\newcommand{\sig}{\mathrm{sig}}

\newcommand{\comment}[1]{}

% Define colors
\def\colorful{1}
\ifnum\colorful=1
\newcommand{\violet}[1]{{\color{violet}{#1}}}
\newcommand{\orange}[1]{{\color{orange}{#1}}}
\newcommand{\blue}[1]{{{\color{blue}#1}}}
\newcommand{\red}[1]{{\color{red} {#1}}}
\newcommand{\green}[1]{{\color{green} {#1}}}
\newcommand{\pink}[1]{{\color{pink}{#1}}}
\newcommand{\gray}[1]{{\color{gray}{#1}}}
\fi
\ifnum\colorful=0
\newcommand{\violet}[1]{{{#1}}}
\newcommand{\orange}[1]{{{#1}}}
\newcommand{\blue}[1]{{{#1}}}
\newcommand{\red}[1]{{{#1}}}
\newcommand{\green}[1]{{{#1}}}
\newcommand{\gray}[1]{{{#1}}}
\fi

\title{Homework 4}
% \author{Tim Randolph}
\date{COMS W3261, Summer B 2021}

\begin{document}

\maketitle

This homework is due \textbf{Monday, 7/26/2021, at 11:59PM EST}. Submit to GradeScope (course code: X3JEX4).

Grading policy reminder: \LaTeX~is preferred, but neatly typed or handwritten solutions are acceptable. Your TAs may dock points for indecipherable writing. Proofs should be complete; that is, include enough information that a reader can clearly tell that the argument is rigorous.

The tool \url{http://madebyevan.com/fsm/} may be useful for drawing finite state machines.

If a question is ambiguous, please state your assumptions. This way, we can give you credit for correct work. (Even better, post on Ed so that we can resolve the ambiguity.)

\clearpage
\section{Problem 1 (10 points)}

\begin{enumerate}
    %\item (4 points). Design a context-free grammar for the following language over the alphabet $\Sigma = \{a, b, \$\}$.
    %\[
    %    L = \{u\$v \; | \; u, v \in \{a, b\}^*, u \text{ is a substring of } v\}.
    %\]
    \item (3 points). Show that the context-free grammar $G_1$, given by the rules
    \begin{align*}
        S &\rightarrow 0A10 \; | \; B10 \\
        B &\rightarrow A0 \; | \; B1 \\
        A &\rightarrow 00 \; | \; \varepsilon
    \end{align*}
    is ambiguous by finding a string with two different leftmost derivations. ($V = \{S, A, B\}$ and $\Sigma=\{0,1\}$.)
    
    \item (3 points). Identify the language generated by $G_1$ above. (It's fine to write this language as a simple regular expression.)
    
    \item (4 points). Design a context-free grammar that generates the same language as $G_2$ but which is not ambiguous (i.e., no string admits two distinct leftmost derivations.) 
\end{enumerate}

\clearpage
\section{Problem 2 (8 points)}
\begin{enumerate}
    \item (3 points). Identify the language generated by the context-free grammar $G_2$ below. (It's fine to describe the language in words or symbols or write a regular expression.)
    \begin{align*}
        S &\rightarrow S\#S \; |\; A \\
        A &\rightarrow 10 \;|\; 0A
    \end{align*}
    
    \item (5 points). Use the procedure outlined in class to convert $G_2$ to an equivalent pushdown automata $P$. (To simplify the state set, you may write a transition function that pushes strings. For example, you might define $\delta(q_1, a, b) = \{(q_2, s)\}$ for some string $s \in \Gamma^*$ as long as you mention that the process of pushing strings implicitly requires some additional states. You may write your answer as a 6-tuple or as an equivalent state diagram. Either way, explain the conversion process.)
\end{enumerate}

\clearpage
\section{Problem 3 (8 points)}


\begin{enumerate}
    \item (3 points). Identify the language generated by the context-free grammar $G_3$ below.
    \begin{align*}
        S &\rightarrow A \; | \; B \\
        A &\rightarrow 0A00 \; | \# \\
        B &\rightarrow 0B000 \; | \#
    \end{align*}
    
    \item (5 points). Prove that the language of $G_3$ is nonregular using the pumping lemma.
\end{enumerate}

\clearpage
\section{Problem 4 (14 points)}


\begin{enumerate}
    \item (3 points). Prove that the class of context-free languages is closed under union. (Hint: We want to show that the union of any two context-free languages is context-free; i.e., for any two languages described by context-free grammars, their union can also be described by a context-free grammar.)
    
    \item (3 points). Prove that the class of context-free languages is closed under concatenation.
    
    \item (3 points). Prove that the class of context-free languages is closed under star.
    
    \item (5 points). Provide a new proof that every regular language is context-free by showing how to convert any regular expression into an equivalent context-free grammar. (Hint: we defined a regular expression inductively using three base cases and three recursive cases corresponding to regular operations. It suffices to explain how each of these cases can be converted into an equivalent CFG.)
\end{enumerate}


\end{document}
