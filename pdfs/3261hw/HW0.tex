\documentclass[letterpaper,11pt,twoside]{article}

\usepackage[left=1in, right=1in, bottom=1.25in, top=1.5in]{geometry}
\usepackage[utf8]{inputenc}
\usepackage{setspace}
%\usepackage{hyperref}
\usepackage{fancyhdr}
\usepackage{amsmath}
\usepackage{amsfonts}
\usepackage{amssymb}
\usepackage{amsthm}
\allowdisplaybreaks
\usepackage[T1]{fontenc}
\usepackage{xcolor}
\usepackage[mathscr]{euscript}
\usepackage{latexsym,bbm,xspace,graphicx,float,mathtools,mathdots,xspace}
\usepackage{enumitem}
\usepackage[ruled,vlined]{algorithm2e}
\usepackage{bm}
\usepackage[backref, colorlinks,citecolor=blue,linkcolor=magenta,bookmarks=true]{hyperref}
\usepackage[nameinlink]{cleveref}

% tikz
\usepackage{subfig}
\usepackage{graphicx}
\usepackage{tikz}

\usepackage{tablefootnote}

\fancypagestyle{plain}{%
\fancyhf{} % clear all header and footer fields
\fancyfoot[C]{\textbf{\thepage}} % except the center
\renewcommand{\headrulewidth}{0pt}
\renewcommand{\footrulewidth}{0pt}}

\theoremstyle{plain}
\newtheorem{theorem}{Theorem}
\newtheorem{assumption}{Assumption}
\newtheorem{corollary}{Corollary}
\newtheorem{lemma}{Lemma}
\newtheorem{conjecture}{Conjecture}
\newtheorem{proposition}{Proposition}
\newtheorem{observation}{Observation}
\newtheorem{claim}{Claim}
\newtheorem{property}{Property}
\newtheorem{op}{Open Problem}
\newtheorem{problem}{Problem}
\newtheorem{question}{Question}

\theoremstyle{definition}
\newtheorem{definition}{Definition}
\newtheorem{example}{Example}
\newtheorem{sketch}{Sketch}
\newtheorem{idea}{Idea}

\theoremstyle{remark}
\newtheorem{remark}{Remark}

\newtheoremstyle{restate}{}{}{\itshape}{}{\bfseries}{~(restated).}{.5em}{\thmnote{#3}}
\theoremstyle{restate}
\newtheorem*{restate}{}



\crefname{theorem}{Theorem}{Theorems}
\crefname{assumption}{Assumption}{Assumptions}
\crefname{corollary}{Corollary}{Corollaries}
\crefname{lemma}{Lemma}{Lemmas}
\crefname{conjecture}{Conjecture}{Conjectures}
\crefname{proposition}{Proposition}{Propositions}
\crefname{observation}{Observation}{Observations}
\crefname{claim}{Claim}{Claims}
\crefname{property}{Property}{Properties}
\crefname{op}{Open Problem}{Open Problems}
\crefname{problem}{Problem}{Problems}
\crefname{question}{Question}{Questions}

% \crefname{fact}{Fact}{Facts}

\crefname{definition}{Definition}{Definitions}
\crefname{example}{Example}{Examples}
\crefname{sketch}{Sketch}{Sketches}
\crefname{idea}{Idea}{Ideas}

% \crefname{condition}{Condition}{Conditions}

\crefname{remark}{Remark}{Remarks}



\crefname{equation}{Equation}{Equations}
\crefname{figure}{Figure}{Figures}
\crefname{table}{Table}{Tables}



%%%%%%%%%%%%%%%%%%%%%%%%%%
% GENERAL-PURPOSE MACROS %
%%%%%%%%%%%%%%%%%%%%%%%%%%
\newcommand{\uth}{\bigskip \bigskip {\huge {\red{UP TO HERE}} \bigskip \bigskip}}
\newcommand{\ignore}[1]{}
\newcommand{\eps}{\epsilon}
\newcommand{\simple}{\mathrm{simple}}
\newcommand{\E}{\operatorname{{\bf E}}}
\newcommand{\Ex}{\mathop{{\bf E}\/}}
\renewcommand{\Pr}{\operatorname{{\bf Pr}}}
\newcommand{\Prx}{\mathop{{\bf Pr}\/}}
\newcommand{\Var}{\operatorname{{\bf Var}}}
\newcommand{\Varx}{\mathop{{\bf Var}\/}}
\newcommand{\tO}{\tilde{O}}
\newcommand{\sgn}{\mathrm{sgn}}
\newcommand{\X}{\mathcal{X}}
\newcommand{\Y}{\mathcal{Y}}
\newcommand{\rZ}{\mathcal{Z}}

\DeclareMathOperator\erf{erf}

\newcommand{\polylog}{\mathrm{polylog}}
\newcommand{\poly}{\mathrm{poly}}
\newcommand{\bcalZ}{\bm{\mathcal{Z}}}
\newcommand{\bx}{\bm{x}}
\newcommand{\by}{\bm{y}}
\newcommand{\bxi}{\bm{\xi}}

%%%%%%%%%%%%%%%%%%
% NUMBER SYSTEMS %
%%%%%%%%%%%%%%%%%%
\newcommand{\R}{\mathbb R}
\newcommand{\RR}{\R_{\geq 0}}
\newcommand{\C}{\mathbb C}
\newcommand{\N}{\mathbb N}
\newcommand{\NN}{\N_{\geq 1}}
\newcommand{\Z}{\mathbb Z}



\renewcommand{\i}{\mathbf{i}}   % for complex numbers
\renewcommand{\d}{\mathrm{d}}   % for integrals
\newcommand{\lhs}{\mathrm{LHS}} % for inequalities
\newcommand{\rhs}{\mathrm{RHS}} % for inequalities
\newcommand{\supp}{\mathrm{supp}}
\renewcommand{\hat}[1]{\widehat{#1}}
\renewcommand{\bar}[1]{\overline{#1}}
\newcommand{\sig}{\mathrm{sig}}



\newcommand{\comment}[1]{}

% Define colors
\def\colorful{1}
\ifnum\colorful=1
\newcommand{\violet}[1]{{\color{violet}{#1}}}
\newcommand{\orange}[1]{{\color{orange}{#1}}}
\newcommand{\blue}[1]{{{\color{blue}#1}}}
\newcommand{\red}[1]{{\color{red} {#1}}}
\newcommand{\green}[1]{{\color{green} {#1}}}
\newcommand{\pink}[1]{{\color{pink}{#1}}}
\newcommand{\gray}[1]{{\color{gray}{#1}}}
\fi
\ifnum\colorful=0
\newcommand{\violet}[1]{{{#1}}}
\newcommand{\orange}[1]{{{#1}}}
\newcommand{\blue}[1]{{{#1}}}
\newcommand{\red}[1]{{{#1}}}
\newcommand{\green}[1]{{{#1}}}
\newcommand{\gray}[1]{{{#1}}}
\fi

\title{COMS 3261 Homework 0}
% \author{Tim Randolph}
\date{Summer B 2021}

\begin{document}

\maketitle

\textbf{This homework set is OPTIONAL. }

\paragraph{Problem 1 (Sets; Sipser 0.1, 0.2, 0.4 and 0.5).}

1.1. Examine the following formal descriptions of sets so that you understand which members they contain. Write a short informal English description of each set.
\begin{enumerate}
    \item $\{1, 3, 5, 7...\}$
    \item $\{n \; | \; n = 2m$ for some $m \in \mathcal{N}\}$
    \item $\{w \; | \; w$ is a string of 0s and 1s and $w$ equals the reverse of $w\}$
\end{enumerate}
1.2. Write formal descriptions of the following sets.
\begin{enumerate}
    \item The set containing the numbers 1, 10, and 100
    \item The set containing all natural numbers less than 5
    \item The set containing nothing at all. (There are two ways to write this: the natural way and using a certain special symbol.)
\end{enumerate}
1.3. If the set $A$ has $a$ elements and $B$ has $b$ elements, how many elements are in $A \times B$? \\
\\
1.4. If $C$ is a set with $c$ elements, how many elements are in the power set of $C$? (Recall that $C$'s power set is the set containing all subsets of $C$.)

\paragraph{Problem 2 (Graphs; Sipser 0.8).}

2.1. Consider the undirected graph $G = (V, E)$, where $V$, the set of nodes, is $\{1, 2, 3, 4\}$ and $E$, the set of edges, is $\{\{1, 2\}, \{2, 3\}, \{1, 3\}, \{2, 4\}, \{1, 4\}\}$. Draw $G$. What are the degrees of each node? Indicate a path from node 3 to node 4 on the graph.

\paragraph{Problem 3 (Proofs; Sipser 0.10, 0.12).}

3.1. Prove that every graph with two or more nodes contains two nodes with the same degree.\\
\\
3.2. Given $f:\mathbb{N} \cup \{0\} \rightarrow \mathbb{N} \cup \{0\}$ where $f$ is defined as: \\
    \[f(x)=
        \begin{cases}
        x+1 & \text{if } x \text{ is even} \\
        x-1 & \text{if } x \text{ is odd} \\
    \end{cases}\]
Prove or provide a counterexample that $f$ is (a) injective, (b) surjective, and (3) bijective. \\ \\
\noindent 3.3. Let $S(n) = 1 + 2 + \cdots + n$ be the sum of the first $n$ natural numbers and let $C(n) = 1^3 + 2^3 + \cdots + n^3$ be the sum of the first $n$ cubes. Prove the following equalities by induction on $n$ to arrive at the curious conclusion that $C(n) = S^2(n)$ for every $n$.
\begin{enumerate}
    \item $S(n) = \frac{1}{2} n (n+1)$
    \item $C(n) = \frac{1}{4}(n^4 + 2n^3 + n^2) = \frac{1}{4}n^2(n+1)^2$.
\end{enumerate} 


\paragraph{Problem 4 (Relations).}

4.1. Let $R$ and $S$ be relations from $A$ to $B$. Prove that: $(R \subseteq S) \rightarrow (R^{-1} \subseteq S^{-1})$.

\paragraph{Problem 5: (Sets and Their Complements).}

Prove that
$$A \cap ((B \cup A^c) \cap B^c ) = \emptyset$$

($A^c$ denotes the complement of $A$ and $B^c$ denotes the complement of $B$.  $\emptyset$ is the empty set.)

\paragraph{Problem 6 (Proof by Contradiction). }

Prove using {\bf contradiction} that for all integers $n$, if 5 divides $n^2$ then 5 divides $n$ (Hint: what does it mean to be not divisible by 5?). 

\paragraph{Problem 7 (Proof by Induction).}
\begin{enumerate}
    \item Assume $n$ is a positive integer. Use induction to prove the following: $$ \frac{1}{1\cdot2} +  \frac{1}{2\cdot3} + \cdot  \frac{1}{n\cdot(n+1)} = 1 - \frac{1}{n+1} $$
    \item Prove that if $m, n \in \mathbb{Z} $ such that $m$ is even and $n$ is odd, then $m + n - 2$ is odd.

\end{enumerate}

\paragraph{Problem 8 (Composition of Functions).}

\begin{enumerate}
    \item   Prove that if $g \circ f $ is one-to-one (injective), then $f$ is one-to-one (injective).
    \item Prove that if $g \circ f$ is onto (surjective), then $g$ is onto (surjective).
\end{enumerate}

\paragraph{Problem 9 (Proof by Construction).}

Prove that for any positive integer $n$, there exists a sequence of n consecutive positive composite integers. [Hint: try to construct such a sequence!]



    

\end{document}

