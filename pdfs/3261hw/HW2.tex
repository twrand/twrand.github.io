\documentclass[letterpaper,11pt,twoside]{article}

\usepackage[left=1in, right=1in, bottom=1.25in, top=1.5in]{geometry}
\usepackage[utf8]{inputenc}
\usepackage{setspace}
%\usepackage{hyperref}
\usepackage{fancyhdr}
\usepackage{amsmath}
\usepackage{amsfonts}
\usepackage{amssymb}
\usepackage{amsthm}
\allowdisplaybreaks
\usepackage[T1]{fontenc}
\usepackage{xcolor}
\usepackage[mathscr]{euscript}
\usepackage{latexsym,bbm,xspace,graphicx,float,mathtools,mathdots,xspace}
\usepackage{enumitem}
\usepackage[ruled,vlined]{algorithm2e}
\usepackage{bm}
\usepackage[backref, colorlinks,citecolor=blue,linkcolor=magenta,bookmarks=true]{hyperref}
\usepackage[nameinlink]{cleveref}

% tikz
\usepackage{subfig}
\usepackage{graphicx}
\usepackage{tikz}

\usepackage{tablefootnote}

\fancypagestyle{plain}{%
\fancyhf{} % clear all header and footer fields
\fancyfoot[C]{\textbf{\thepage}} % except the center
\renewcommand{\headrulewidth}{0pt}
\renewcommand{\footrulewidth}{0pt}}

\theoremstyle{plain}
\newtheorem{theorem}{Theorem}
\newtheorem{assumption}{Assumption}
\newtheorem{corollary}{Corollary}
\newtheorem{lemma}{Lemma}
\newtheorem{conjecture}{Conjecture}
\newtheorem{proposition}{Proposition}
\newtheorem{observation}{Observation}
\newtheorem{claim}{Claim}
\newtheorem{property}{Property}
\newtheorem{op}{Open Problem}
\newtheorem{problem}{Problem}
\newtheorem{question}{Question}

\theoremstyle{definition}
\newtheorem{definition}{Definition}
\newtheorem{example}{Example}
\newtheorem{sketch}{Sketch}
\newtheorem{idea}{Idea}

\theoremstyle{remark}
\newtheorem{remark}{Remark}

\newtheoremstyle{restate}{}{}{\itshape}{}{\bfseries}{~(restated).}{.5em}{\thmnote{#3}}
\theoremstyle{restate}
\newtheorem*{restate}{}



\crefname{theorem}{Theorem}{Theorems}
\crefname{assumption}{Assumption}{Assumptions}
\crefname{corollary}{Corollary}{Corollaries}
\crefname{lemma}{Lemma}{Lemmas}
\crefname{conjecture}{Conjecture}{Conjectures}
\crefname{proposition}{Proposition}{Propositions}
\crefname{observation}{Observation}{Observations}
\crefname{claim}{Claim}{Claims}
\crefname{property}{Property}{Properties}
\crefname{op}{Open Problem}{Open Problems}
\crefname{problem}{Problem}{Problems}
\crefname{question}{Question}{Questions}

% \crefname{fact}{Fact}{Facts}

\crefname{definition}{Definition}{Definitions}
\crefname{example}{Example}{Examples}
\crefname{sketch}{Sketch}{Sketches}
\crefname{idea}{Idea}{Ideas}

% \crefname{condition}{Condition}{Conditions}

\crefname{remark}{Remark}{Remarks}



\crefname{equation}{Equation}{Equations}
\crefname{figure}{Figure}{Figures}
\crefname{table}{Table}{Tables}



%%%%%%%%%%%%%%%%%%%%%%%%%%
% GENERAL-PURPOSE MACROS %
%%%%%%%%%%%%%%%%%%%%%%%%%%
\newcommand{\uth}{\bigskip \bigskip {\huge {\red{UP TO HERE}} \bigskip \bigskip}}
\newcommand{\ignore}[1]{}
\newcommand{\eps}{\epsilon}
\newcommand{\simple}{\mathrm{simple}}
\newcommand{\E}{\operatorname{{\bf E}}}
\newcommand{\Ex}{\mathop{{\bf E}\/}}
\renewcommand{\Pr}{\operatorname{{\bf Pr}}}
\newcommand{\Prx}{\mathop{{\bf Pr}\/}}
\newcommand{\Var}{\operatorname{{\bf Var}}}
\newcommand{\Varx}{\mathop{{\bf Var}\/}}
\newcommand{\tO}{\tilde{O}}
\newcommand{\sgn}{\mathrm{sgn}}
\newcommand{\X}{\mathcal{X}}
\newcommand{\Y}{\mathcal{Y}}
\newcommand{\rZ}{\mathcal{Z}}

\DeclareMathOperator\erf{erf}

\newcommand{\polylog}{\mathrm{polylog}}
\newcommand{\poly}{\mathrm{poly}}
\newcommand{\bcalZ}{\bm{\mathcal{Z}}}
\newcommand{\bx}{\bm{x}}
\newcommand{\by}{\bm{y}}
\newcommand{\bxi}{\bm{\xi}}

%%%%%%%%%%%%%%%%%%
% NUMBER SYSTEMS %
%%%%%%%%%%%%%%%%%%
\newcommand{\R}{\mathbb R}
\newcommand{\RR}{\R_{\geq 0}}
\newcommand{\C}{\mathbb C}
\newcommand{\N}{\mathbb N}
\newcommand{\NN}{\N_{\geq 1}}
\newcommand{\Z}{\mathbb Z}

\renewcommand{\i}{\mathbf{i}}   % for complex numbers
\renewcommand{\d}{\mathrm{d}}   % for integrals
\newcommand{\lhs}{\mathrm{LHS}} % for inequalities
\newcommand{\rhs}{\mathrm{RHS}} % for inequalities
\newcommand{\supp}{\mathrm{supp}}
\renewcommand{\hat}[1]{\widehat{#1}}
\renewcommand{\bar}[1]{\overline{#1}}
\newcommand{\sig}{\mathrm{sig}}

\newcommand{\comment}[1]{}

% Define colors
\def\colorful{1}
\ifnum\colorful=1
\newcommand{\violet}[1]{{\color{violet}{#1}}}
\newcommand{\orange}[1]{{\color{orange}{#1}}}
\newcommand{\blue}[1]{{{\color{blue}#1}}}
\newcommand{\red}[1]{{\color{red} {#1}}}
\newcommand{\green}[1]{{\color{green} {#1}}}
\newcommand{\pink}[1]{{\color{pink}{#1}}}
\newcommand{\gray}[1]{{\color{gray}{#1}}}
\fi
\ifnum\colorful=0
\newcommand{\violet}[1]{{{#1}}}
\newcommand{\orange}[1]{{{#1}}}
\newcommand{\blue}[1]{{{#1}}}
\newcommand{\red}[1]{{{#1}}}
\newcommand{\green}[1]{{{#1}}}
\newcommand{\gray}[1]{{{#1}}}
\fi

\title{Homework 2}
% \author{Tim Randolph}
\date{COMS W3261, Summer B 2021}

\begin{document}

\maketitle

This homework is due \textbf{Monday, 7/12/2021, at 11:59PM EST}. Submit to GradeScope (course code: X3JEX4).

Grading policy reminder: \LaTeX~is preferred, but neatly typed or handwritten solutions are acceptable. Your TAs may dock points for indecipherable writing. Proofs should be complete; that is, include enough information that a reader can clearly tell that the argument is rigorous.

Remember that the tool \url{http://madebyevan.com/fsm/} may be useful for drawing finite state machines.

\clearpage
\section{Problem 1 (7 points)}

Evaluate each of the following regular expressions and write down the language it describes as a set or with a single sentence. (Example: $01^+ = \{w \; | \; w $ consists of a single 0 followed by at least one 1$\}$ or ``This regular expression describes the language of strings that consist of a single 0 followed by at least one 1''.)

\begin{enumerate}
    \item (1 point.) Let $\Sigma = \{0,1\}$. Evaluate $0\Sigma^*1\Sigma^*0$.
    \item (1 point.) Evaluate $(01 \cup 0 \cup 1)0^*(1^+)\emptyset$.
\end{enumerate}

Write regular expressions that evaluate to the languages given.
\begin{enumerate}[resume]
    \item (1 point.)
    \[
        \{w \; | \; w \text{ consists of a substring of }a\text{'s and }b\text{'s of even length, followed by the substring `01'.}\}
    \]
    \item (1 point.) 
    \[
        \{ w \; | \; w \text{ is a string of 0's with length divisible by } 2, 3, 5, \text{ or } 7.\}
    \]
\end{enumerate}
Evaluate each of the following regular expressions and write down the language it describes as a set or with a single sentence.
\begin{enumerate}[resume]
    \item (1 point.) Let $H = \{0, 1, 2, 3, 4, 5, 6, 7, 8, 9, A, B, C, D, E, F\}$. Evaluate $\#HHHHHH$.
    \item (1 point.) Let $P = \{K, Q, R, B, N\}$, $X = \{a, b, c, d, e, f, g, h\}$, $Y = \{1, 2, 3, 4, 5, 6, 7, 8\}$. Evaluate $(P \cup \varepsilon)(\times \cup \varepsilon)XY$. 
    \item (1 point.) Evaluate the regular expression
    \[
        \rightarrow(\bigcirc(\underset{0}{\rightarrow} \cup \underset{1}{\rightarrow}))^*\odot.
    \]
    (The alphabet is $\Sigma = \{\rightarrow, \underset{0}{\rightarrow}, \underset{1}{\rightarrow}, \bigcirc, \odot\}$.)
\end{enumerate}

\clearpage
\section{Problem 2 (6 points)}

\begin{enumerate}
    \item (6 points). Draw a state diagram for an NFA \textbf{with at most 3 states} that recognizes the regular expression
        \[
            ((1 \circ 10) \cup 11)^*.
        \]
        Explain in words why your NFA recognizes the language specified.
\end{enumerate}

\clearpage
\section{Problem 3 (6 points)}
    \begin{enumerate}
        \item (6 points.) Given a language $A$, define the language pre$(A)$ as follows:
        \[
            \text{pre}(A) := \{xy \; | x \in A\}.
        \]
        Prove that the class of regular languages is closed under pre$(\cdot)$.
    \end{enumerate}

\clearpage
\section{Problem 4 (1 point)}
    \begin{enumerate}
        \item What is one thing the instructor or course staff could do better to make the material or expectations clearer or more convenient for you?
        
        %\blue{(Any coherent response.)}
        
        \item Adjusting for the compressed timeframe of this course (two weeks in one), how are you finding the difficulty so far in relation to other CUCS courses? (For example: `much easier', `somewhat easier', `about the same', `somewhat harder', `much harder'.) 
        
        
        \item (Optional) Any other thoughts? Thank you!
    \end{enumerate}

\end{document}
