\documentclass[letterpaper,11pt,twoside]{article}

\usepackage[left=1in, right=1in, bottom=1.25in, top=1.5in]{geometry}
\usepackage[utf8]{inputenc}
\usepackage{setspace}
%\usepackage{hyperref}
\usepackage{fancyhdr}
\usepackage{amsmath}
\usepackage{amsfonts}
\usepackage{amssymb}
\usepackage{amsthm}
\allowdisplaybreaks
\usepackage[T1]{fontenc}
\usepackage{xcolor}
\usepackage[mathscr]{euscript}
\usepackage{latexsym,bbm,xspace,graphicx,float,mathtools,mathdots,xspace}
\usepackage{enumitem}
\usepackage[ruled,vlined]{algorithm2e}
\usepackage{bm}
\usepackage[backref, colorlinks,citecolor=blue,linkcolor=magenta,bookmarks=true]{hyperref}
\usepackage[nameinlink]{cleveref}

% tikz
\usepackage{subfig}
\usepackage{graphicx}
\usepackage{tikz}

\usepackage{tablefootnote}

\fancypagestyle{plain}{%
\fancyhf{} % clear all header and footer fields
\fancyfoot[C]{\textbf{\thepage}} % except the center
\renewcommand{\headrulewidth}{0pt}
\renewcommand{\footrulewidth}{0pt}}

\theoremstyle{plain}
\newtheorem{theorem}{Theorem}
\newtheorem{assumption}{Assumption}
\newtheorem{corollary}{Corollary}
\newtheorem{lemma}{Lemma}
\newtheorem{conjecture}{Conjecture}
\newtheorem{proposition}{Proposition}
\newtheorem{observation}{Observation}
\newtheorem{claim}{Claim}
\newtheorem{property}{Property}
\newtheorem{op}{Open Problem}
\newtheorem{problem}{Problem}
\newtheorem{question}{Question}

\theoremstyle{definition}
\newtheorem{definition}{Definition}
\newtheorem{example}{Example}
\newtheorem{sketch}{Sketch}
\newtheorem{idea}{Idea}

\theoremstyle{remark}
\newtheorem{remark}{Remark}

\newtheoremstyle{restate}{}{}{\itshape}{}{\bfseries}{~(restated).}{.5em}{\thmnote{#3}}
\theoremstyle{restate}
\newtheorem*{restate}{}



\crefname{theorem}{Theorem}{Theorems}
\crefname{assumption}{Assumption}{Assumptions}
\crefname{corollary}{Corollary}{Corollaries}
\crefname{lemma}{Lemma}{Lemmas}
\crefname{conjecture}{Conjecture}{Conjectures}
\crefname{proposition}{Proposition}{Propositions}
\crefname{observation}{Observation}{Observations}
\crefname{claim}{Claim}{Claims}
\crefname{property}{Property}{Properties}
\crefname{op}{Open Problem}{Open Problems}
\crefname{problem}{Problem}{Problems}
\crefname{question}{Question}{Questions}

% \crefname{fact}{Fact}{Facts}

\crefname{definition}{Definition}{Definitions}
\crefname{example}{Example}{Examples}
\crefname{sketch}{Sketch}{Sketches}
\crefname{idea}{Idea}{Ideas}

% \crefname{condition}{Condition}{Conditions}

\crefname{remark}{Remark}{Remarks}



\crefname{equation}{Equation}{Equations}
\crefname{figure}{Figure}{Figures}
\crefname{table}{Table}{Tables}



%%%%%%%%%%%%%%%%%%%%%%%%%%
% GENERAL-PURPOSE MACROS %
%%%%%%%%%%%%%%%%%%%%%%%%%%
\newcommand{\uth}{\bigskip \bigskip {\huge {\red{UP TO HERE}} \bigskip \bigskip}}
\newcommand{\ignore}[1]{}
\newcommand{\eps}{\epsilon}
\newcommand{\simple}{\mathrm{simple}}
\newcommand{\E}{\operatorname{{\bf E}}}
\newcommand{\Ex}{\mathop{{\bf E}\/}}
\renewcommand{\Pr}{\operatorname{{\bf Pr}}}
\newcommand{\Prx}{\mathop{{\bf Pr}\/}}
\newcommand{\Var}{\operatorname{{\bf Var}}}
\newcommand{\Varx}{\mathop{{\bf Var}\/}}
\newcommand{\tO}{\tilde{O}}
\newcommand{\sgn}{\mathrm{sgn}}
\newcommand{\X}{\mathcal{X}}
\newcommand{\Y}{\mathcal{Y}}
\newcommand{\rZ}{\mathcal{Z}}

\DeclareMathOperator\erf{erf}

\newcommand{\polylog}{\mathrm{polylog}}
\newcommand{\poly}{\mathrm{poly}}
\newcommand{\bcalZ}{\bm{\mathcal{Z}}}
\newcommand{\bx}{\bm{x}}
\newcommand{\by}{\bm{y}}
\newcommand{\bxi}{\bm{\xi}}

% Blind (markerless) footnotes
\newcommand\blfootnote[1]{%
  \begingroup
  \renewcommand\thefootnote{}\footnote{#1}%
  \addtocounter{footnote}{-1}%
  \endgroup
}

%%%%%%%%%%%%%%%%%%
% NUMBER SYSTEMS %
%%%%%%%%%%%%%%%%%%
\newcommand{\R}{\mathbb R}
\newcommand{\RR}{\R_{\geq 0}}
\newcommand{\C}{\mathbb C}
\newcommand{\N}{\mathbb N}
\newcommand{\NN}{\N_{\geq 1}}
\newcommand{\Z}{\mathbb Z}

\renewcommand{\i}{\mathbf{i}}   % for complex numbers
\renewcommand{\d}{\mathrm{d}}   % for integrals
\newcommand{\lhs}{\mathrm{LHS}} % for inequalities
\newcommand{\rhs}{\mathrm{RHS}} % for inequalities
\newcommand{\supp}{\mathrm{supp}}
\renewcommand{\hat}[1]{\widehat{#1}}
\renewcommand{\bar}[1]{\overline{#1}}
\newcommand{\sig}{\mathrm{sig}}

\newcommand{\comment}[1]{}

% Define colors
\def\colorful{1}
\ifnum\colorful=1
\newcommand{\violet}[1]{{\color{violet}{#1}}}
\newcommand{\orange}[1]{{\color{orange}{#1}}}
\newcommand{\blue}[1]{{{\color{blue}#1}}}
\newcommand{\red}[1]{{\color{red} {#1}}}
\newcommand{\green}[1]{{\color{green} {#1}}}
\newcommand{\pink}[1]{{\color{pink}{#1}}}
\newcommand{\gray}[1]{{\color{gray}{#1}}}
\fi
\ifnum\colorful=0
\newcommand{\violet}[1]{{{#1}}}
\newcommand{\orange}[1]{{{#1}}}
\newcommand{\blue}[1]{{{#1}}}
\newcommand{\red}[1]{{{#1}}}
\newcommand{\green}[1]{{{#1}}}
\newcommand{\gray}[1]{{{#1}}}
\fi

\title{Homework 3}
% \author{Tim Randolph}
\date{COMS W3261, Summer A 2022}

\begin{document}

\maketitle

This homework is due \textbf{Monday, 6/12/2023, at 11:59pm EST}. Submit to GradeScope (course code: K3VK75). If you use late days, the absolute latest we can accept a submission is Friday at 11:59 PM EST.

\textbf{Grading policy reminder:} \LaTeX~is preferred, but neatly typed or handwritten solutions are acceptable.\footnote{The website \href{https://www.overleaf.com/}{Overleaf} (essentially Google Docs for LaTeX) may make compiling and organizing your .tex files easier. Here's a quick \href{https://www.overleaf.com/learn/latex/Learn_LaTeX_in_30_minutes}{tutorial}.} Feel free to use the .tex file for the homework as a template to write up your answers, or use the template posted on the course website. Your TAs may dock points for indecipherable writing.\\

Proofs should be complete; that is, include enough information that a reader can clearly tell that the argument is rigorous. \\

If a question is ambiguous, please state your assumptions. This way, we can give you credit for correct work. (Even better, post on Ed so that we can resolve the ambiguity.) \\

\textbf{\LaTeX~resources.}
\begin{itemize}
    \item \href{https://detexify.kirelabs.org/classify.html}{Detexify} is a nice tool that lets you draw a symbol and returns the \LaTeX~codes for similar symbols. 
    \item The tool \href{https://www.tablesgenerator.com/}{Table Generator} makes building tables in \LaTeX~much easier.
    \item The tool \href{http://madebyevan.com/fsm/}{Finite State Machine Designer} may be useful for drawing automata. See also this example (\href{https://static.us.edusercontent.com/files/HZeTXimODzWeLvHIqsvjL2BG}{PDF}) (\href{https://static.us.edusercontent.com/files/RI3W8tQNvHMWFe9MkXV1KztA}{.tex}) of how to make fancy edges (courtesy of Eumin Hong).
    \item The website \href{https://www.mathcha.io/}{mathcha.io} allows you to draw diagrams and convert them to \LaTeX~code.
    \item To use the previous drawing tools (and for most drawing in \LaTeX), you'll need to use the package Tikz (add the command ``\textbackslash usepackage\{tikz\}'' to the preamble of your .tex file to import the package). 
    \item \href{https://www.overleaf.com/learn/latex/Positioning_of_Figures}{This tutorial} is a helpful guide to positioning figures.
\end{itemize}  



\clearpage
\section{Problem 1 (14 points)}
    \begin{enumerate}
         \item (3 points). What is the language of the grammar $G_1$ below? Express this language as a set, with a sentence, or as a simple regular expression and explain your reasoning.

        (Note that here we use the abbreviated or `rules-only' way to write a grammar. The variables $\{S, A, B, C\}$ can be read off the lefthand side, and the terminals $\{0, 1, 2, 3\}$ are the remaining symbols.)
        \begin{align*}
            S &\rightarrow 0A \; | \; 1B \; | \; 2C \\
            A &\rightarrow 1B \; | \; 2C    \\
            B &\rightarrow 2C \; \\
            C &\rightarrow 3 \\
        \end{align*}

        \item (2 points). How does the language of $G_1$ change if we add the rule $S \rightarrow SS$ to the grammar?

    
        \item (3 points). What is the language of the grammar $G_2$ below? Express this language as a set, with a sentence, or as a simple regular expression and explain your reasoning.
        
        \begin{align*}
            S &\rightarrow AB \\
            A &\rightarrow 01A10 \; | \; 0 \\
            B &\rightarrow 1B \; | \; \epsilon \\
        \end{align*}
        
        
        \item (3 points). Design a grammar for the language
        \[
            D = \{ 1^n 0^{2m} 1^m 0^{2n} \; | \; m, n \geq 2 \}
        \]
        and explain why your grammar produces $D$. (This language includes such strings as $110000110000$.) You may use the brief representation of grammars (i.e., rules-only) or write out the full 4-tuple.
        
        \item (3 points). Design a grammar for the language represented by the regular expression
        \[
            R_1 = 1^+ \cup 0^+ \cup 1^*01^*
        \]
        and explain why your grammar produces the same language. You may use the brief representation of grammars (i.e., rules-only) or write-out the full 4-tuple.
    \end{enumerate}

    \blfootnote{ Rationale: The goal of this question is to practice interpreting and building context-free grammars. }
    \blfootnote{ References: Sipser p. 102 and Lightning Review 5 (CFG definition and deriving strings), Sipser p.105 (figuring out the language of a CFG), and Sipser p.106-107 (tips for building CFGs). }

% Stopped here

\clearpage
\section{Problem 2 (12 points)}

\begin{enumerate}
    \item (6 points.) Prove that the language
    \[
        A = \{a^i b^j c^k \; | \; i + j \geq k\}
    \]
    over the alphabet $\Sigma = \{a,b, c\}$ is nonregular using the pumping lemma.
    
    \item (6 points.) Prove that the language
    \[
        B = \{ a^i b^j c^k \; | \; i < j \text{ OR } i > k; \text{ also } i, j, k \geq 1 \}
    \]
    over the alphabet $\Sigma = \{a, b, c\}$ is nonregular using the pumping lemma.
    
    \blfootnote{ Rationale: The goal of this question is to practice using the pumping lemma to show that languages are nonregular. }
    \blfootnote{ References: Sipser p. 78-79 and Lightning Review 4 (the pumping lemma), Sipser p.80-82 and Lightning Review 5 (using the pumping lemma). }
\end{enumerate}

    
\clearpage
\section{Problem 3 (6 points)}
\begin{enumerate}
    \item (5 points.) We've proved that the regular languages are closed under regular operations such as complement, union, concatenation and star: if we apply these operations to regular languages, we get a regular language. However, we have not proved that the \emph{non}regular languages are closed under the regular operations. 
    
    Suppose $A$ and $B$ are nonregular languages. Is their union $A \cup B$ guaranteed to be nonregular? If so, provide a proof. If not, provide a counterexample.

    \item (1 point.) Give an example of a language $C$ such that $C \cap A$ is regular for any language $A$, whether $A$ is regular or not. (You should be able to do this for an arbitrary alphabet $\Sigma$, but feel free to assume $\Sigma = \{0,1\}$ for this question if you would like.)

\end{enumerate}

    \blfootnote{ Rationale: The goal of this question is to practice reasoning about closure properties and (non)regularity. }
    \blfootnote{ References: Languages that we've previously proved to be nonregular (refer to previous questions in this HW, or the Lecture 4 notes) which can serve as examples. Recall that a regular language is any language recognized by some DFA; or equivalently, recognized by some NFA; or equivalently, expressed by some regular expression. A nonregular language is any language that \emph{can't} be recognized/expressed in this way. }

\clearpage
\section{Problem 4 (8 points) }

\begin{enumerate}
    \item (4 points.) Consider the following ``proof'' of nonregularity, which contains a logical error:

    \begin{enumerate}
        \item Consider the language $A = \{x \in \{0,1\}^* \; | \; |x| \text{ is divisible by 3. } \}$. We'll assume for contradiction that $A$ is regular.
        \item By the pumping lemma, under our assumption there exists some number $p$ such that every string $s \in A$ with $s \geq p$ can be divided into 3 substrings $x$, $y$, and $z$ such that (1) $xy^iz$ is in the language for all $i \geq 0$, (2) $|y| > 0$ and (3) $|xy| \leq p$.
        \item We'll choose the contradiction string $0^{3p}$, which has length $3p > p$ and is in the language. We'll show that it fails the conditions of the pumping lemma.
        \item To satisfy condition (2), it must be true that $y$ is a string containing at least one $0$.
        \item Consider $y = 00$. In this case, the string $xy^2z$ has length $3p + 2$.
        \item Since $|xyz| = 3p$ is divisible by 3, $|xy^2z| = 3p + 2$ is not divisible by 3 and thus $x y^2 z$ is not in the language. 
        \item Thus the string $0^{3p}$ cannot be pumped, which is a contradiction. Therefore $A$ is nonregular.
    \end{enumerate}

    This can't be right: $A$ is a regular language, as we've already seen. In what step does the error occur? Why is this proof invalid?
    
    \item (4 points.) The following proof also contains a logical error. In what step does it occur, and why is this proof invalid?

    \begin{enumerate}
        \item Consider the language $B = \{a^i b^j c^k \; | \; i \leq j \text{ OR } j < k \text{ OR } k < i\}$. We'll assume for contradiction that $B$ is regular.
        \item By the pumping lemma, under our assumption there exists some number $p$ such that every string $s \in A$ with $s \geq p$ can be divided into 3 substrings $x$, $y$, and $z$ such that (1) $xy^iz$ is in the language for all $i \geq 0$, (2) $|y| > 0$ and (3) $|xy| \leq p$.
        \item We'll choose the contradiction string $a^{p} b^{p} c^{p}$. which has length $3p > p$. Moreover, $a^p b^p c^p$ is in the language because the number of $a$'s is less than or equal to the number of $b$'s. We'll show that it fails the conditions of the pumping lemma.
        \item To satisfy conditions (2) and (3), it must be true that $y$ is a string containing at least one $a$, and only $a$'s.
        \item Note that $a^p b^p c^p$ is in the language because it satisfies the first of the three OR'ed conditions in the language definition: if we set $i, j, k = p$, it is true that $i \leq j$ but not true that $j < k$ or $k < i$.
        \item Now consider the string $x y^2 z = a^{p + |y|} b^p c^p$. Since it is no longer true that $i \leq j$, this string is not in the language.
        \item Thus the string $a^p b^p c^p$ cannot be pumped, which is a contradiction. Therefore $B$ is nonregular.
    \end{enumerate}
\end{enumerate}

\blfootnote{ Rationale: The goal of this question is to practice the pumping lemma from a different angle: common issues that arise when working through PL proofs. }
    \blfootnote{ References: Sipser p. 78-79 and Lightning Review 4 (the pumping lemma), Sipser p.80-82 and Lightning Review 5 (using the pumping lemma).  }


\clearpage
\section{Problem 5 (1 bonus point)}

The \href{https://en.wikipedia.org/wiki/Myhill\%E2\%80\%93Nerode_theorem}{Myhill-Nerode theorem} says that a language $L$ is a regular language if and only if $L$ has a finite number of \href{https://en.wikipedia.org/wiki/Equivalence_class}{equivalence classes}  (i.e., $L$ would not be a regular language if it had an infinite number of equivalence classes). Because it gives us an `if and only if' condition, it's more powerful than the pumping lemma.
\newline

Consider the language $L$, defined over the alphabet $\Sigma=\{0, 1\}$:
\[
    E=\{w \; | \; w \text{ starts or ends with the substring } 0\}.
\]

Is $E$ regular or nonregular? Prove your claim using the Myhill-Nerode theorem. (Hint: You should define all the equivalence classes for $E$ in terms of distinguishing extensions, or prove that there are an infinite number of equivalence classes under this relation.)\\


\blfootnote{ Rationale: Optional, just for fun. The bonus point will add 1 to your total score on this HW, which is out of 40. }
\blfootnote{ Resources: Wiki on the \href{https://en.wikipedia.org/wiki/Myhill\%E2\%80\%93Nerode_theorem}{Myhill-Nerode theorem} and \href{https://en.wikipedia.org/wiki/Equivalence_class}{equivalence classes}. }
\end{document}
